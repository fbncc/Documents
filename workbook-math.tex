\documentclass[leqno,autodetect-engine,dvipdfmx-if-dvi,ja=standard,a4paper,12pt]{bxjsbook}
%\documentclass{book} %% make4htを使うときはこの行のコメントを外し,上の行をコメントにする.

%%%%%%%%%%%%%%%%%%%%%% preamble 20200217-00
\usepackage{subfiles}


% expand the figure environment
\usepackage{subfigure}


% tikz 

\usepackage{tikz}
\usetikzlibrary{decorations.pathmorphing}

\usepackage{tkz-euclide}
%\usepackage{tkz-fct}
\usetikzlibrary{arrows}

%\usetkzobj{all}

\usepackage{pgfplots}

\usepackage{bussproofs}
\usepackage{braket}

% AMSMATH
\usepackage{amsmath, amssymb, amsthm, amsfonts}
\usepackage{mathrsfs,amsxtra}
\usepackage{mathtools}
\mathtoolsset{showonlyrefs=true} % 参照したときのみ,行番号を振る。

% arc
%\usepackage{fourier-otf}

% proof
\usepackage{bussproofs}
\usepackage{braket}



% HyperRef
%\usepackage[pdfencoding=auto]{hyperref}

% frame line
\usepackage{tcolorbox}

% 索引の作成 (headerに入れるとlatexmkでエラーが発生するので削除)
%\usepackage{makeidx}
%\makeindex

%% tableofcontents level
\setcounter{tocdepth}{1}

% 定理環境の設定
\swapnumbers
\theoremstyle{definition}
\newtheorem{thm}{}[section]
\newtheorem{ihr}[thm]{Ir}
\newtheorem{axm}[thm]{Ax}
\newtheorem{dfn}[thm]{Df}

\newtheorem{eg}[thm]{EG} % 例
\newtheorem{qst}[thm]{Q} % 問い
\newtheorem{exm}[thm]{Ex} %問題

\newtheorem{nb}[thm]{NB} % nota bene (latin) 注意せよ。
\newtheorem{cf}[thm]{CF}  % confero 参照せよ。比較せよ。

% proof enviornment QED
\renewcommand{\qedsymbol}{\hfill QED \par}


% subsection
\setcounter{secnumdepth}{4}
\renewcommand{\thesubsection}{(\arabic{subsection})}
\renewcommand{\thesubsubsection}{(\alph{subsubsection})}
%\renewcommand{\theparagraph}{(\roman{paragraph})}
%\renewcommand{\thesubparagraph}{(\roman{subparagraph})}

% list  1) 2) ...  a) b)
\def\labelenumi{\theenumi)}
\def\labelenumii{\theenumii)}
\def\labelenumiii{\theenumiii)}
\def\labelenumiv{\theenumiv)}

% 略記号
\newcommand{\etal}{\emph{et al}.}
\newcommand{\ie}{\emph{i.e.,}}
\newcommand{\as}{\emph{a.e.}}
%\renewcommand{\eg}{\emph{e.g.,}}



% 論理記号
\newcommand{\ldfn}{:\iff}
\newcommand{\edfn}{:=}

\renewcommand{\iff}{\Leftrightarrow}         % 同値
\renewcommand{\implies}{\Rightarrow}       % ならば
\providecommand{\lxor}{\veebar}                % 排他的論理和
\providecommand{\lfalse}{\bot}                   % 偽
\providecommand{\ltre}{\top}                      % 真


% 集合記号
\newcommand{\SetZero}{\{0\}} % 零集合

\renewcommand{\P}[1]{\mathfrak{P} #1}     % ベキ集合

\newcommand{\eps}{\varepsilon}

% 写像記号
\newcommand{\hooklongrightarrow}{\lhook\joinrel\longrightarrow}
\newcommand{\mapmono}{\hooklongrightarrow}
\newcommand{\hooklongleftarrow}{\longleftarrow\joinrel\rhook}

\newcommand{\twoheadlongrightarrow}{\relbar\joinrel\twoheadrightarrow}
\newcommand{\mapepi}{\twoheadlongrightarrow}
\newcommand{\twoheadlongleftarrow}{\twoheadleftarrow\joinrel\relbar}

\newcommand{\twoheadhooklongrightarrow}{\lhook\joinrel\twoheadlongrightarrow}
\newcommand{\mapiso}{\twoheadhooklongrightarrow}
\newcommand{\twoheadhooklongleftarrow}{\twoheadlongleftarrow\joinrel\rhook}

\newcommand{\MAP}[3]{#1 : #2 \longrightarrow #3}
\newcommand{\MAPISO}[3]{#1 : #2 \mapiso #3}
\newcommand{\MAPEPI}[3]{#1 : #2 \mapepi #3}
\newcommand{\MAPMONO}[3]{#1 : #2 \mapmono #3}

\newcommand{\map}{\longrightarrow}



\newcommand{\vecb}[1]{\boldsymbol{#1}}
\newcommand{\veca}[1]{\overrightarrow{#1}}

\newcommand{\seg}[1]{\overline{#1}}


\newcommand{\Sp}[1]{sp(#1)}



% mathmatical Symbol
\DeclareMathOperator{\Hom}{\mathrm{Hom}}
\DeclareMathOperator{\Iso}{\mathrm{Iso}}
\DeclareMathOperator{\Mor}{\mathrm{Mor}}
\DeclareMathOperator{\Ob}{\mathrm{Ob}}


\DeclareMathOperator{\Dom}{\mathrm{Dom}}
\DeclareMathOperator{\Cod}{\mathrm{Cod}}
\DeclareMathOperator{\Img}{\mathrm{Img}}

\DeclareMathOperator{\Image}{\mathrm{Im}}

\DeclareMathOperator{\Tar}{\mathrm{Tar}}
\DeclareMathOperator{\Src}{\mathrm{Src}}

\DeclareMathOperator{\Succ}{\mathrm{Succ}}
\DeclareMathOperator{\Prev}{\mathrm{Prev}}

\DeclareMathOperator{\pr}{\mathrm{pr}}

\DeclareMathOperator{\Int}{\mathrm{Int}}

\DeclareMathOperator{\lcm}{\mathrm{lcm}}

\DeclareMathOperator{\sgn}{\mathrm{sgn}}

\DeclareMathOperator{\uniq}{\mathrm{uniq}}

\DeclareMathOperator{\rank}{\mathrm{rank}}

\DeclareMathOperator{\mesh}{\mathrm{mesh}}


\newcommand{\abs}[1]{\left\lvert#1\right\rvert}%絶対値 
\newcommand{\norm}[1]{\left\lVert#1\right\rVert}%ノルム
%\DeclarePairedDelimiter\abs{\lvert}{\rvert}%
%\DeclarePairedDelimiter\norm{\lVert}{\rVert}%
\DeclarePairedDelimiter\series{\{}{\}}%


\newcommand{\E}{\mathbb{E}}
\renewcommand{\L}{\mathbb{L}}

\newcommand{\N}{\mathbb{N}}
\newcommand{\NP}{\mathbb{N}^{+}}
\newcommand{\Z}{\mathbb{Z}}
\newcommand{\ZP}{\mathbb{Z}^{+}}
\newcommand{\ZN}{\mathbb{Z}^{-}}
\newcommand{\Q}{\mathbb{Q}}
\newcommand{\QP}{\mathbb{Q}^{+}}
\newcommand{\QN}{\mathbb{Q}^{-}}
\newcommand{\R}{\mathbb{R}}
\newcommand{\RP}{\mathbb{R}^{+}}
\newcommand{\RN}{\mathbb{R}^{-}}
\renewcommand{\C}{\mathbb{C}}




% matrix symbols
\newcommand{\mA}{\boldsymbol{A}}
\newcommand{\mB}{\boldsymbol{B}}
\newcommand{\mC}{\boldsymbol{C}}
\newcommand{\mD}{\boldsymbol{D}}
\newcommand{\mE}{\boldsymbol{E}}
\newcommand{\mF}{\boldsymbol{F}}
\newcommand{\mG}{\boldsymbol{G}}
\newcommand{\mH}{\boldsymbol{H}}
\newcommand{\mI}{\boldsymbol{I}}
\newcommand{\mJ}{\boldsymbol{J}}
\newcommand{\mK}{\boldsymbol{K}}
\newcommand{\mL}{\boldsymbol{L}}
\newcommand{\mM}{\boldsymbol{M}}
\newcommand{\mN}{\boldsymbol{N}}
\newcommand{\mO}{\boldsymbol{O}}
\newcommand{\mP}{\boldsymbol{P}}
\newcommand{\mQ}{\boldsymbol{Q}}
\newcommand{\mR}{\boldsymbol{R}}
\newcommand{\mS}{\boldsymbol{S}}
\newcommand{\mT}{\boldsymbol{T}}
\newcommand{\mU}{\boldsymbol{U}}
\newcommand{\mV}{\boldsymbol{V}}
\newcommand{\mW}{\boldsymbol{W}}
\newcommand{\mX}{\boldsymbol{X}}
\newcommand{\mY}{\boldsymbol{Y}}
\newcommand{\mZ}{\boldsymbol{Z}}

% vector symbols
\newcommand{\vA}{\boldsymbol{A}}
\newcommand{\vB}{\boldsymbol{B}}
\newcommand{\vC}{\boldsymbol{C}}
\newcommand{\vD}{\boldsymbol{D}}
\newcommand{\vE}{\boldsymbol{E}}
\newcommand{\vF}{\boldsymbol{F}}
\newcommand{\vG}{\boldsymbol{G}}
\newcommand{\vH}{\boldsymbol{H}}
\newcommand{\vI}{\boldsymbol{I}}
\newcommand{\vJ}{\boldsymbol{J}}
\newcommand{\vK}{\boldsymbol{K}}
\newcommand{\vL}{\boldsymbol{L}}
\newcommand{\vM}{\boldsymbol{M}}
\newcommand{\vN}{\boldsymbol{N}}
\newcommand{\vO}{\boldsymbol{O}}
\newcommand{\vP}{\boldsymbol{P}}
\newcommand{\vQ}{\boldsymbol{Q}}
\newcommand{\vR}{\boldsymbol{R}}
\newcommand{\vS}{\boldsymbol{S}}
\newcommand{\vT}{\boldsymbol{T}}
\newcommand{\vU}{\boldsymbol{U}}
\newcommand{\vV}{\boldsymbol{V}}
\newcommand{\vW}{\boldsymbol{W}}
\newcommand{\vX}{\boldsymbol{X}}
\newcommand{\vY}{\boldsymbol{Y}}
\newcommand{\vZ}{\boldsymbol{Z}}

\newcommand{\va}{\boldsymbol{a}}
\newcommand{\vb}{\boldsymbol{b}}
\newcommand{\vc}{\boldsymbol{c}}
\newcommand{\vd}{\boldsymbol{d}}
\newcommand{\ve}{\boldsymbol{e}}
\newcommand{\vf}{\boldsymbol{f}}
\newcommand{\vg}{\boldsymbol{g}}
\newcommand{\vh}{\boldsymbol{h}}
\newcommand{\vi}{\boldsymbol{i}}
\newcommand{\vj}{\boldsymbol{j}}
\newcommand{\vk}{\boldsymbol{k}}
\newcommand{\vl}{\boldsymbol{l}}
\newcommand{\vm}{\boldsymbol{m}}
\newcommand{\vn}{\boldsymbol{n}}
\newcommand{\vo}{\boldsymbol{o}}
\newcommand{\vp}{\boldsymbol{p}}
\newcommand{\vq}{\boldsymbol{q}}
\newcommand{\vr}{\boldsymbol{r}}
\newcommand{\vs}{\boldsymbol{s}}
\newcommand{\vt}{\boldsymbol{t}}
\newcommand{\vu}{\boldsymbol{u}}
\newcommand{\vv}{\boldsymbol{v}}
\newcommand{\vw}{\boldsymbol{w}}
\newcommand{\vx}{\boldsymbol{x}}
\newcommand{\vy}{\boldsymbol{y}}
\newcommand{\vz}{\boldsymbol{z}}


\newcommand{\cA}{\mathcal{A}}
\newcommand{\cB}{\mathcal{B}}
\newcommand{\cC}{\mathcal{C}}
\newcommand{\cD}{\mathcal{D}}
\newcommand{\cE}{\mathcal{E}}
\newcommand{\cF}{\mathcal{F}}
\newcommand{\cG}{\mathcal{G}}
\newcommand{\cH}{\mathcal{H}}
\newcommand{\cI}{\mathcal{I}}
\newcommand{\cJ}{\mathcal{J}}
\newcommand{\cK}{\mathcal{K}}
\newcommand{\cL}{\mathcal{L}}
\newcommand{\cM}{\mathcal{M}}
\newcommand{\cN}{\mathcal{N}}
\newcommand{\cO}{\mathcal{O}}
\newcommand{\cP}{\mathcal{P}}
\newcommand{\cQ}{\mathcal{Q}}
\newcommand{\cR}{\mathcal{R}}
\newcommand{\cS}{\mathcal{S}}
\newcommand{\cT}{\mathcal{T}}
\newcommand{\cU}{\mathcal{U}}
\newcommand{\cV}{\mathcal{V}}
\newcommand{\cW}{\mathcal{W}}
\newcommand{\cX}{\mathcal{X}}
\newcommand{\cY}{\mathcal{Y}}
\newcommand{\cZ}{\mathcal{Z}}

%%%%%%%%%%%%%%%%%%%%%% preamble end


% latex to html をする場合,次の4行のコメントを外す.
%\usepackage{alternative4ht}
%	\altusepackage{fontspec}
%	\altusepackage{xeCJK}
%	\altusepackage{xunicode}

\begin{document}
	\tableofcontents
	\part{Latex}
		\chapter{Tex4ht}
			Latexからhtmlに変換するAppを使って見た.
			\section{日本語を扱うための準備}
				\subsection{条件}
					次の条件が必要です.
					\begin{enumerate}
						\item Lualatexを使う.
						\item helpers4htを使う.
					\end{enumerate}
				
					Lualatexは多分インストールされているので,helpers4ht4ht ( \url{https://github.com/michal-h21/helpers4ht} )
					をインストールする.
					インストール方法は,ググって調べました.私は,ここ \url{https://qiita.com/toyolab/items/12ac4fb95fcf211ccf62} を参考にしました.
			
				\subsection{Latexファイルの修正}
					日本語が入ったlatexソースファイルをそのままTex4htが扱えないので,次の2点についてソースファイルを修正する.
						\begin{enumerate}
							\item documentclass を book にする.
							\item alternative4ht 他のパッケージを読み込む.
						\end{enumerate}
					例えば,次の例のように,latex用のdocumentclass文とTex4ht用のdocumentclassの2つを用意して,htmlに変換するときだけコメントを外すようにしている.
						\begin{verbatim}
								%\documentclass[leqno,autodetect-engine,dvipdfmx-if-dvi,ja=standard,a4paper,12pt]{bxjsbook}
								\documentclass{book} %% make4htを使うときはこの行のコメントを外し,上の行をコメントにする.
								
								% latex to html をする場合,次の4行のコメントを外す.
								\usepackage{alternative4ht}
									\altusepackage{fontspec}
									\altusepackage{xeCJK}
									\altusepackage{xunicode}
								
								\begin{document}
									本文
								\end{document}
						\end{verbatim}
			\section{htmlへの変換}
				Latexソースファイルにエラーがなければ,次のコマンド
				\begin{quote}
					make4ht -l  ほげほげ.tex "index=2,3"
				\end{quote}
			を入力すれば,htmlファイルができる.
			
			-l はluatexを使うフラグで,"index=2,3" はhtmlを章ごとにhtmlファイルを作成する指示です.
			
			詳しくは,マニュアル
	\part{git /cdot gihub}
	\part{実数}
	\part{微分積分}
		\chapter{三角函数}
			\section{公式}
				\begin{thm}[加法定理]
						\begin{align}
							\cos(x+y) &= \cos(x)\cos(y) - \sin(x)\sin(y) \\
							\sin(x+y) &= \sin(x) \cos(y) + \cos(x)\sin(y) \\
							\tan(x+y) &= \frac{\tan(x) + \tan(y)}{1 - \tan(x)\tan(y)}
						\end{align}
				\end{thm}
			
				\begin{thm}[ピタゴラスの基本三角函数公式]
					\begin{equation}
						\cos^2(x) + \sin^2(x) = 1
					\end{equation}
				\end{thm}
			
				\begin{thm} $\sin(x)$ の $x=0$ のときの微分係数
					\begin{equation}
						\left. \frac{d}{dx} \sin(x)\right\lvert_{x=0} = \lim_{h \to 0} \frac{ \sin(h)} {h} = 1
					\end{equation}
				\end{thm}
			
				\begin{thm}[倍角の公式]
					\begin{align}
						\cos(2x) &= \cos^2(x) - \sin^2(x) \\
						              &= 2 \cos^2(x) -1 \\
						              &= 1 -2 \sin^2(x) \\
						\sin(2x) &= 2\sin(x)\cos(x) \\
						\tan(2x) &= \frac{2\tan(x)} {1 - \tan^2(x)} 
					\end{align}
				\end{thm}
			
				\begin{thm}[半角の公式]
					\begin{align}
						\cos^2 \left( \frac{x}{2} \right) &= \frac{1 + \cos(x)}{2} \\
						\sin^2 \left( \frac{x}{2} \right) &= \frac{1 - \cos(x)}{2} \\
						\tan^2 \left( \frac{x}{2} \right) &= \frac{1 - \cos(x)} {1 + \cos(x)}
					\end{align}
				\end{thm}
			
				\begin{thm}[和積公式]
					\begin{align}
						\cos(x) + \cos(y) &= 2 \sin \left( \frac{x+y}{2} \right) \cos \left( \frac{x+y}{2} \right) \\
						\cos(x) - \cos(y) &= -2 \sin \left( \frac{x+y}{2} \right) \cos \left( \frac{x+y}{2} \right) \\
						\sin(x) + \sin(y) &= 2 \cos \left( \frac{x+y}{2} \right) \cos \left( \frac{x+y}{2} \right) 
					\end{align}
				\end{thm}
			
				\begin{thm}[積和公式]
						\begin{align}
							\cos(x) \cos(y) &= \frac{1}{2} (\cos(x+y) + \cos(x-y)) \\
							\sin(x) \sin(y) &= -\frac{1}{2} (\cos(x+y) - \cos(x-y)) \\
							\sin(x) \cos(y) &=  \frac{1}{2} (\sin(x+y) + \sin(x-y))
						\end{align}
				\end{thm}
			
				\begin{thm}[負角公式]
					\begin{align}
						\cos(-x) &= \cos(x) \\
						\sin(-x) &= - \sin(x) \\
						\tan(-x) &= - \tan(x)
					\end{align}
				\end{thm}
			
				\begin{thm}[余角公式]
					\begin{align}
						\cos\left( \frac{\pi}{2} - x \right) & = \cos(x) \\
						\sin\left( \frac{\pi}{2} - x \right) & = \sin(x) \\
						\tan\left( \frac{\pi}{2} - x \right) & = \cot(x)
					\end{align}
				\end{thm}
				
				\begin{thm}[補角公式]
					\begin{align}
						\cos(\pi -x) &= -\cos(x) \\
						\sin(\pi-x) &= \sin(x) \\
						\tan(\pi-x) &= - \tan(x)
					\end{align}
				\end{thm}
	\part{線形代数}
		\chapter{行列式}
			\section{集合}
				\begin{dfn} 自然数 $n$ の切片 $\N[n]$ を,次のように定義する.
					\begin{equation}
						\N(n) \edfn \{ x \in \N ; 0 \le x < n \}
					\end{equation}
				\end{dfn}
				\begin{dfn} 集合$A$が有限集合とは,ある自然数$n$の切片$\N(n)$から有限集合$A$への全単射が存在することである.すなわち,
					\begin{equation}
						\exists n \in \N ; \exists \varphi : \N(n) \map A; \varphi : \text{全単射}
					\end{equation}
					である.
				\end{dfn}
			\section{置換}
				\subsection{置換の定義}
				\subsection{あみだくじ}
				\subsection{置換の性質}
				\subsection{偶置換と奇置換}
				\subsection{$\sgn$ 函数}
			\section{実数の四則演算}
					
					\subsubsection{数列}
					\begin{dfn}[有限列] ある自然数 $n$ から実数の集合 $\R$ への写像$a$を実数の有限列といい,$\{ a_{i} \}_{i=0}^{n-1}$で表す.すなわち,
						\begin{equation}
							\{a_i\}_{i=0}^{n-1} \ldfn a: \N(n) \map \R
						\end{equation}
					\end{dfn}
					\begin{dfn}[無限列] 自然数の集合 $\N$ から実数の集合 $\R$ への写像$a$を実数の有限列といい,$\{ a_{i} \}_{i=0}^{\infty}$で表す.すなわち,
						\begin{equation}
							\{a_i\}_{i=0}^{\infty} \ldfn a: \N \map \R
						\end{equation}
					\end{dfn}
				
					\subsubsection{総和}
						\begin{dfn} 数列 $\{a_i\}_{i=0}$ の総和を次のように定義する.
							\begin{equation}
								\sum_{i=0}^{n} a_i \edfn \begin{cases}
									a_0 & n = 0 \\
									\sum_{i=0}^{n-1} a_i + a_{n} &  0 < n
								\end{cases}
							\end{equation}
						\end{dfn}
					
					\begin{dfn} $l,n$が自然数のとき,数列 $\{a_i\}_{i=0}^{\infty}$ の $l$ 項から $l + n$ 項までの総和を次のように定義する.
						\begin{equation}
							\sum_{i=l}^{l+n} a_i \edfn \sum_{i=0}^{n} a_{l+i}
						\end{equation}
					\end{dfn}
					
					\begin{thm}[結合法則] 数列 $\{a_i\}_{i=0}^{n}$ について,次の一般結合法則が成り立つ.
						\begin{equation}
							0 \le \forall k < n ; \sum_{i=0}^{k} a_i + \sum_{i=k+1}^{n} a_i = \sum_{i=0}^{n} a_i
						\end{equation} 
					\end{thm}
					
					\begin{thm}[交換法則] 数列 $\{a_i\}_{i=0}^{\infty}$ について,次の一般交換法則が成り立つ.
						\begin{equation}
							\forall \sigma \in S_n ; \sum_{i=0}^{n} a_{\sigma(i)}  = \sum_{i=0}^{n} a_i
						\end{equation} 
					\end{thm}
					\subsubsection{総乗}
					\begin{dfn} 数列 $\{a_i\}_{i=0}^{n}$ の総乗を次のように定義する.
						\begin{equation}
							\prod_{i=0}^{n} a_i \edfn \begin{cases}
								a_0 & n = 0 \\
								\prod_{i=0}^{n-1} a_i + a_{n} &  0 < n
							\end{cases}
						\end{equation}
					\end{dfn}

			\section{行列式の定義}
				\begin{dfn}[行列式] $n$次正方行列 $\vA=(a_{i j})$ の行列式 $\det(\vA)$ を次のように定義する.
					\begin{equation}
						\det(\vA) \edfn \sum_{\sigma \in S_n} \sgn(\sigma) \prod_{j=1}^{n} a_{\sigma(j) j}
					\end{equation}
					行列式 $\det(\vA)$ を,
					\begin{equation}
						\det(a_{i j})
					\end{equation}
					\begin{equation}
						\begin{vmatrix}
							a_{1 1} & a_{1 2} & \dots & a_{1 n} \\
							a_{2 1} & a_{2 2} & \dots & a_{2 n} \\
							\vdots & \vdots & \dots & \vdots \\
							a_{n 1} & a_{n 2} & \dots & a_{n n}
						\end{vmatrix}
					\end{equation}
					と表すことがある.さらに,行列 $\vA$ を列ベクトル $\va_1,\va_2,\dots,\va_n$ で表すとき,
					\begin{equation}
						\det({\va_1,\va_2,\dots,\va_n})
					\end{equation}
					と表すことがある.
				\end{dfn}
			
				\begin{thm} 転置行列 $ ^{t}\vA$ の行列式 $\det( ^{t}\vA)$ は,元の行列 $\vA$ の行列式 $\det(\vA)$ と等しい.すなわち,
					\begin{equation}
						\det( ^{t}\vA) = \det{\vA} 
					\end{equation}
					が成り立つ.
				\end{thm}

				\section{行列式の展開}
				
					\begin{thm}
						\begin{equation}
							\begin{vmatrix}
								a_{1 1} & a_{1 2} & \dots & a_{1 n} \\
								a_{2 1} & a_{2 2} & \dots & a_{2 n} \\
								\vdots & \vdots & \dots & \vdots \\
								a_{n 1} & a_{n 2} & \dots & a_{n n}
							\end{vmatrix}
							= \begin{vmatrix}
								a_{1 1} & 0 & \dots & 0 \\
								0 & a_{2 2} & \dots & a_{2 n} \\
								\vdots & \vdots & \dots & \vdots \\
								0 & a_{n 2} & \dots & a_{n n}
							\end{vmatrix}
							+ \begin{vmatrix}
								0 & a_{1 2} & \dots & 0 \\
								a_{2 1} & 0 & \dots & a_{2 n} \\
								\vdots & \vdots & \dots & \vdots \\
								a_{n 1} & 0 & \dots & a_{n n}
							\end{vmatrix}
							+ \dots + \begin{vmatrix}
								0 & \dots & 0 & a_{1 n} \\
								a_{2 1} & \dots & a_{2 n-1} & 0 \\
								\vdots & \dots & \vdots & \vdots \\
								a_{n 1} & \dots & a_{n n-1} & 0
							\end{vmatrix}
						\end{equation}
					\end{thm}
					\begin{proof}
						\begin{align}
							\begin{vmatrix}
								a_{1 1} & a_{1 2} & \dots & a_{1 n} \\
								a_{2 1} & a_{2 2} & \dots & a_{2 n} \\
								\vdots & \vdots & \dots & \vdots \\
								a_{n 1} & a_{n 2} & \dots & a_{n n}
							\end{vmatrix}
							&= \begin{vmatrix}
								a_{1 1} + 0 & a_{1 2} & \dots & a_{1 n} \\
								0 + a_{2 1} & a_{2 2} & \dots & a_{2 n} \\
								\vdots & \vdots & \dots & \vdots \\
								0 + a_{n 1} & a_{n 2} & \dots & a_{n n}
							\end{vmatrix} \\
							&= \begin{vmatrix}
								a_{1 1} & a_{1 2} & \dots & a_{1 n} \\
								0  & a_{2 2} & \dots & a_{2 n} \\
								\vdots & \vdots & \dots & \vdots \\
								0 & a_{n 2} & \dots & a_{n n}
							\end{vmatrix} 
							+
							\begin{vmatrix}
								0 & a_{1 2} & \dots & a_{1 n} \\
								a_{2 1} & a_{2 2} & \dots & a_{2 n} \\
								\vdots & \vdots & \dots & \vdots \\
								a_{n 1} & a_{n 2} & \dots & a_{n n}
							\end{vmatrix} \\
							&= \begin{vmatrix}
								a_{1 1} & 0 & \dots & 0 \\
								0  & a_{2 2} & \dots & a_{2 n} \\
								\vdots & \vdots & \dots & \vdots \\
								0 & a_{n 2} & \dots & a_{n n}
							\end{vmatrix} 
							+
							\begin{vmatrix}
								0 & a_{1 2} & \dots & a_{1 n} \\
								a_{2 1} & a_{2 2} & \dots & a_{2 n} \\
								\vdots & \vdots & \dots & \vdots \\
								a_{n 1} & a_{n 2} & \dots & a_{n n}
							\end{vmatrix}
						\end{align}
					
						\begin{align}
							\begin{vmatrix}
								0 & a_{1 2} & a_{1 3} & \dots & a_{1 n} \\
								a_{2 1} & a_{2 2} & a_{2 3} & \dots & a_{2 n} \\
								\vdots & \vdots & \vdots & \dots & \vdots \\
								a_{n 1} & a_{n 2} & a_{n 3} & \dots & a_{n n}
							\end{vmatrix}
							&= \begin{vmatrix}
								0 & a_{1 2} + 0 & a_{1 3} & \dots & a_{1 n} \\
								a_{2 1} & 0+ a_{2 2} & a_{2 3} & \dots & a_{2 n} \\
								\vdots & \vdots & \vdots & \dots & \vdots \\
								a_{n 1} & 0 + a_{n 2} & a_{n 3} & \dots & a_{n n}
							\end{vmatrix} \\
							&= \begin{vmatrix}
								0 & a_{1 2} & a_{1 3} & \dots & a_{1 n} \\
								a_{2 1}  & 0 & a_{2 3}& \dots & a_{2 n} \\
								\vdots & \vdots & \vdots & \dots & \vdots \\
								a_{n 1} & 0 & a_{n 3} & \dots & a_{n n}
							\end{vmatrix} 
							+
							\begin{vmatrix}
								0 & 0  & a_{1 3} & \dots & a_{1 n} \\
								a_{2 1} & a_{2 2} & a_{2 3} & \dots & a_{2 n} \\
								\vdots & \vdots & \vdots & \dots & \vdots \\
								a_{n 1} & a_{n 2} & a_{n 3} & \dots & a_{n n}
							\end{vmatrix} \\
							&= \begin{vmatrix}
								0 & a_{1 2} & a_{1 3} & \dots & 0 \\
								a_{2 1}  & 0 & a_{2 3} & \dots & a_{2 n} \\
								\vdots & \vdots & \dots & \vdots \\
								a_{n 1} & 0 & a_{n 3} & \dots & a_{n n}
							\end{vmatrix} 
							+
							\begin{vmatrix}
								0 & 0 & a_{1 3} &\dots & a_{1 n} \\
								a_{2 1} & a_{2 2} & a_{2 3} & \dots & a_{2 n} \\
								\vdots & \vdots & \vdots & \dots & \vdots \\
								a_{n 1} & a_{n 2} & a_{n 3} & \dots & a_{n n}
							\end{vmatrix}
						\end{align}
					
						\begin{equation}
							\begin{vmatrix}
								0 &   \dots & 0 & a_{1 n} \\
								a_{2 1} & \dots & a_{2 n-1} & a_{2 n} \\
								\vdots & \dots & \vdots & \vdots \\
								a_{n 1} & \dots & a_{n n-1}  & a_{n n}
							\end{vmatrix}
							= \begin{vmatrix}
								0 &   \dots & 0 & a_{1 n} \\
								a_{2 1} & \dots & a_{2 n-1} & 0 \\
								\vdots & \dots & \vdots & \vdots \\
								a_{n 1} & \dots & a_{n n-1}  & 0
							\end{vmatrix}
						\end{equation}
					\end{proof}
\end{document}
