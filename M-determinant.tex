\documentclass[leqno,autodetect-engine, dvipdfmx-if-dvi,ja=standard]{bxjsarticle}
\input{header.tex}
\title{行列式}
\begin{document}
	
	\section{数列}
	
	\section{総和 $\sum$}
	
		\begin{dfn}[総和] 数列 $\Set{a_i}$ の $a_1$ から $a_n$ までの総和 $\sum$ を帰納的に定義する.
			\begin{align}
				\sum_{i=1}^{1} a_{i} &:= a_1 \\
				\sum_{i=1}^{n+1} a_i &:= \left( \sum_{i=1}^{n} a_i \right) + a_{n+1}
			\end{align}
		\end{dfn}
	
		\begin{eg} 総和を括弧 $( \  )$ で加法演算 $+$ の実行順序を表すと
			\begin{equation}
				\sum_{i=1}^n a_i = ((\dots((a_1) + a_2)+ \dots )+ a_{n-1}) + a_n
			\end{equation}
			となる.括弧の役割は,加法演算の実行順序を表す.実行順序を加法 $+$ への下付き数字で表すことにより,括弧をはずしてもよい.すなわち
			\begin{equation}
				a_1 +_{1} a_2 +_{2} \dots +_{n-2} \ \ a_{n-1} +_{n-1} a_n = \ ((\dots((a_1) + a_2)+ \dots )+ a_{n-1}) + a_n
			\end{equation}
			が成り立っている.
			
			例えば,加法の結合法則は,.
			\begin{equation}
				a + (b + c) = (a + b) + c
			\end{equation}
			であるから,この加法の実行順序を表すと
			\begin{equation} 
				a +_2 b +_1 c= a +_1 b +_2 c
			\end{equation}
			今,置換 $\sigma$ を
			\begin{equation}
				\sigma = 
				\begin{pmatrix}
					1 & 2 \\
					2 & 1 
				\end{pmatrix}
			\end{equation}
			とすれば,結合法則は
			\begin{equation} 
				a +_{\sigma(1)} \ \ b +_{\sigma(2)} \ \ c= a +_1 b +_2 c
			\end{equation}
			と表すことができる.
			
			$n$ 項の結合法則は隣り合う加法演算の順序を交換することにより,加法演算の順序が左から順番に演算することに等しいことができる.すなわち,
			\begin{equation}
				a_1 \ +_{\sigma(1)} \ a_2 \ +_{\sigma(2)}  \dots +_{\sigma(n-1)} \ a_n = a_1 \ +_1 \ a_2 \ +_2 \dots +_{n-1} \ a_n 
			\end{equation}
			が成り立つ.これは,任意の置換が隣接互換の積で表せることで\part{title}
		\end{eg}
	
		\begin{thm}[一般の結合法則] $n$ 項の加法は,加法の順番によらず,一定の値になる.すなわち,
			\begin{equation}
				a_1 \ +_{\sigma(1)} \ a_2 \ +_{\sigma(2)}  \dots +_{\sigma(n-1)} \ a_n = a_1 \ +_1 \ a_2 \ +_2 \dots +_{n-1} \ a_n 
			\end{equation}
		\end{thm}
	
		\begin{thm}[一般の交換法則]
		\end{thm}
	
		\begin{dfn} 自然数の集合 $\N_1^n$ から集合 $\Lambda$ への全単射 $\varphi :  \N_1^n \map \Lambda$ が存在するとき, $\Lambda$ を添字集合とする数列 $\Set{a_\lambda}_{\lambda \in \Lambda} $ の総和 $\sum_{\lambda \in \Lambda} a_{\lambda}$ を次のように定義する.
			\begin{equation}
				\sum_{\lambda \in \Lambda} a_{\lambda} := \sum_{i=1}^n a_{\varphi(i)}
			\end{equation}
		\end{dfn}
	
	\section{行列式の定義}
	
		\begin{dfn} 行列 $\mA = (\va_1, \dots,\va_n) = (a_{i j})$ とするとき,この行列 $\mA$ の行列式 $\det(\mA)$ を次のように定義する.
			\begin{equation}
				\det(\mA) := \sum_{\sigma \in S_n} \sgn(\sigma) \ a_{1 \, \sigma(1)} \dots a_{n \, \sigma(n)}
			\end{equation}
		
			行列式は,次のように表すことがある.
			\begin{equation}
				\det(\va_1,\dots,\va_n), \det(a_{i j}), \rvert A \lvert,  \rvert \va_1, \dots, \va_n \lvert, \rvert a_{i j} \lvert,
				\begin{vmatrix}
					a_{1 1} & a_{1 2} & \dots & a_{1 n} \\
					a_{2 1} & a_{2 2} & \dots & a_{2 n } \\
					           &             & \dots &              \\
					a_{n 1} & a_{n 2} & \dots & a_{n n}					           
				\end{vmatrix}
			\end{equation}
		\end{dfn}
	

		
	\section{行列式の双線形性}
		\begin{thm} 双線形性
			\begin{enumerate}
				\item a
					\begin{equation}
						\det(\va_1, \dots, \va_j' + \va_j'', \dots, \va_n) = \det(\va_1, \dots, \va_j', \dots, \va_n) + \det(\va_1, \dots, \va_j'', \dots, \va_n)  
					\end{equation}
				
					\begin{equation}
						\begin{vmatrix}
							a_{1 1} & \dots & a_{1 j}' + a_{1 j}''  & \dots & a_{1 n} \\
							           & \dots &   \dots                & \dots &           \\
							a_{n 1} & \dots & a_{n j}' + a_{n j}'' & \dots & a_{n n}
						\end{vmatrix} 
						=
						\begin{vmatrix}
							a_{1 1} & \dots & a_{1 j}' & \dots & a_{1 n} \\
							& \dots &   \dots                & \dots &           \\
							a_{n 1} & \dots & a_{n j}' & \dots & a_{n n}
						\end{vmatrix} 
						+
						\begin{vmatrix}
							a_{1 1} & \dots & a_{1 j}''  & \dots & a_{1 n} \\
							& \dots &   \dots                & \dots &           \\
							a_{n 1} & \dots &  a_{n j}'' & \dots & a_{n n}
						\end{vmatrix}
					\end{equation}
				\item b
					\begin{equation}
						\det(\va_1, \dots, \lambda \va_j, \dots, \va_n) = \lambda \det(\va_1, \dots, \va_j, \dots, \va_n)
					\end{equation}
					\begin{equation}
						\begin{vmatrix}
							a_{1 1} & \dots & \lambda a_{1 j} & \dots & a_{1 n} \\
							& \dots &   \dots                & \dots &           \\
							a_{n 1} & \dots & \lambda a_{n j}  & \dots & a_{n n}
						\end{vmatrix} 
						= \lambda
						\begin{vmatrix}
							a_{1 1} & \dots & a_{1 j} & \dots & a_{1 n} \\
							& \dots &   \dots                & \dots &           \\
							a_{n 1} & \dots & a_{n j} & \dots & a_{n n}
						\end{vmatrix} 
					\end{equation}
			\end{enumerate}
		\end{thm}

		\begin{thm} $\tau \in S_n$
			\begin{equation}
				\det(\va_{\tau(1)}, \dots, \va_{\tau(j)}, \dots, \va_{\tau(n)}) = \sgn(\tau) \det(\va_1, \dots, \va_j, \dots, \va_n)
			\end{equation}
			\begin{equation}
				\begin{vmatrix}
					a_{1 \tau(1)} & \dots & \lambda a_{1 \tau(j)} & \dots & a_{1 \tau(n)} \\
					& \dots &   \dots                & \dots &           \\
					a_{n \tau(1)} & \dots & \lambda a_{n \tau(j)}  & \dots & a_{n \tau(n)}
				\end{vmatrix} 
				= \sgn(\tau)
				\begin{vmatrix}
					a_{1 1} & \dots & a_{1 j} & \dots & a_{1 n} \\
					& \dots &   \dots                & \dots &           \\
					a_{n 1} & \dots & a_{n j} & \dots & a_{n n}
				\end{vmatrix} 
			\end{equation}
		\end{thm}

		\begin{thm}
			\begin{equation}
				\det({}^t\!\mA) = \det(\mA)
			\end{equation}
		\end{thm}

	
	\section{行列式の展開}
		\begin{dfn}
			\begin{equation}
				\mA_{i j} = \begin{pmatrix}
						a_{1 1} & a_{1 2} & \dots & a_{1 j} & \dots & a_{1 n} \\
						a_{2 1} & a_{2 2} & \dots & a_{2 j} & \dots & a_{2 n} \\
						            &            & \dots &      &     & \\
					   a_{i 1} & a_{i 2} & \dots & a_{i j} & \dots & a_{i n} \\
						            &            & \dots &      &     & \\
						a_{n 1} & a_{n 2} & \dots & a_{n j} & \dots & a_{n n} \\          
					\end{pmatrix}
			\end{equation}
		\end{dfn}
	
		\begin{thm}
			\begin{equation}
				\det(\mA) = a_{i 1} \det(\mA_{1 i}) + \dots + a_{i j} \det(\mA_{i j}) + \dots + a_{i n} \det(\mA_{i n})
			\end{equation}
		\end{thm}

\end{document}