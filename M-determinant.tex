\documentclass[leqno,autodetect-engine, dvipdfmx-if-dvi,ja=standard]{bxjsarticle}
%%%%%%%%%%%%%%%%%%%%%% preamble 20200217-00
\usepackage{subfiles}


% expand the figure environment
\usepackage{subfigure}


% tikz 

\usepackage{tikz}
\usetikzlibrary{decorations.pathmorphing}

\usepackage{tkz-euclide}
%\usepackage{tkz-fct}
\usetikzlibrary{arrows}

%\usetkzobj{all}

\usepackage{pgfplots}

\usepackage{bussproofs}
\usepackage{braket}

% AMSMATH
\usepackage{amsmath, amssymb, amsthm, amsfonts}
\usepackage{mathrsfs,amsxtra}
\usepackage{mathtools}
%\mathtoolsset{showonlyrefs=true} % 参照したときのみ,行番号を振る。

% arc
%\usepackage{fourier-otf}

% proof
\usepackage{bussproofs}
\usepackage{braket}



% HyperRef
\usepackage[pdfencoding=auto]{hyperref}

% frame line
\usepackage{tcolorbox}

% 索引の作成 (headerに入れるとlatexmkでエラーが発生するので削除)
%\usepackage{makeidx}
%\makeindex

%% tableofcontents level
\setcounter{tocdepth}{1}

% 定理環境の設定
\swapnumbers
\theoremstyle{definition}
\newtheorem{thm}{}[section]
\newtheorem{ihr}[thm]{Ir}
\newtheorem{axm}[thm]{Ax}
\newtheorem{dfn}[thm]{Df}

\newtheorem{eg}[thm]{EG} % 例
\newtheorem{qst}[thm]{Q} % 問い
\newtheorem{exm}[thm]{Ex} %問題

\newtheorem{nb}[thm]{NB} % nota bene (latin) 注意せよ。
\newtheorem{cf}[thm]{CF}  % confero 参照せよ。比較せよ。

% proof enviornment QED
\renewcommand{\qedsymbol}{\hfill QED \par}


% subsection
\setcounter{secnumdepth}{4}
\renewcommand{\thesubsection}{(\arabic{subsection})}
\renewcommand{\thesubsubsection}{(\alph{subsubsection})}
%\renewcommand{\theparagraph}{(\roman{paragraph})}
%\renewcommand{\thesubparagraph}{(\roman{subparagraph})}

% list  1) 2) ...  a) b)
\def\labelenumi{\theenumi)}
\def\labelenumii{\theenumii)}
\def\labelenumiii{\theenumiii)}
\def\labelenumiv{\theenumiv)}

% mathematics symbols epi mono iso
\newcommand{\hooklongrightarrow}{\lhook\joinrel\longrightarrow}
\newcommand{\mapmono}{\hooklongrightarrow}
\newcommand{\hooklongleftarrow}{\longleftarrow\joinrel\rhook}

\newcommand{\twoheadlongrightarrow}{\relbar\joinrel\twoheadrightarrow}
\newcommand{\mapepi}{\twoheadlongrightarrow}
\newcommand{\twoheadlongleftarrow}{\twoheadleftarrow\joinrel\relbar}

\newcommand{\twoheadhooklongrightarrow}{\lhook\joinrel\twoheadlongrightarrow}
\newcommand{\mapiso}{\twoheadhooklongrightarrow}
\newcommand{\twoheadhooklongleftarrow}{\twoheadlongleftarrow\joinrel\rhook}

\newcommand{\MAP}[3]{#1 : #2 \longrightarrow #3}
\newcommand{\MAPISO}[3]{#1 : #2 \mapiso #3}
\newcommand{\MAPEPI}[3]{#1 : #2 \mapepi #3}
\newcommand{\MAPMONO}[3]{#1 : #2 \mapmono #3}

\newcommand{\map}{\longrightarrow}

\renewcommand{\P}[1]{\mathfrak{P} #1}

\newcommand{\vecb}[1]{\boldsymbol{#1}}
\newcommand{\veca}[1]{\overrightarrow{#1}}

\newcommand{\seg}[1]{\overline{#1}}


\newcommand{\Sp}[1]{sp(#1)}

\renewcommand{\iff}{\Leftrightarrow}
\renewcommand{\implies}{\Rightarrow}
\providecommand{\lxor}{\veebar}
\providecommand{\lfalse}{\bot}
\providecommand{\ltre}{\top}

% mathmatical Symbol
\DeclareMathOperator{\Hom}{\mathrm{Hom}}
\DeclareMathOperator{\Iso}{\mathrm{Iso}}
\DeclareMathOperator{\Mor}{\mathrm{Mor}}
\DeclareMathOperator{\Ob}{\mathrm{Ob}}


\DeclareMathOperator{\Dom}{\mathrm{Dom}}
\DeclareMathOperator{\Cod}{\mathrm{Cod}}
\DeclareMathOperator{\Img}{\mathrm{Img}}

\DeclareMathOperator{\Image}{\mathrm{Im}}

\DeclareMathOperator{\Tar}{\mathrm{Tar}}
\DeclareMathOperator{\Src}{\mathrm{Src}}

\DeclareMathOperator{\Succ}{\mathrm{Succ}}
\DeclareMathOperator{\Prev}{\mathrm{Prev}}

\DeclareMathOperator{\pr}{\mathrm{pr}}

\DeclareMathOperator{\Int}{\mathrm{Int}}

\DeclareMathOperator{\lcm}{\mathrm{lcm}}

\DeclareMathOperator{\sgn}{\mathrm{sgn}}

\DeclareMathOperator{\uniq}{\mathrm{uniq}}

\DeclareMathOperator{\rank}{\mathrm{rank}}

\DeclareMathOperator{\mesh}{\mathrm{mesh}}


\newcommand{\abs}[1]{\left\lvert#1\right\rvert}%絶対値 
\newcommand{\norm}[1]{\left\lVert#1\right\rVert}%ノルム
%\DeclarePairedDelimiter\abs{\lvert}{\rvert}%
%\DeclarePairedDelimiter\norm{\lVert}{\rVert}%
\DeclarePairedDelimiter\series{\{}{\}}%


\newcommand{\E}{\mathbb{E}}
\renewcommand{\L}{\mathbb{L}}

\newcommand{\N}{\mathbb{N}}
\newcommand{\NP}{\mathbb{N}^{+}}
\newcommand{\Z}{\mathbb{Z}}
\newcommand{\ZP}{\mathbb{Z}^{+}}
\newcommand{\ZN}{\mathbb{Z}^{-}}
\newcommand{\Q}{\mathbb{Q}}
\newcommand{\QP}{\mathbb{Q}^{+}}
\newcommand{\QN}{\mathbb{Q}^{-}}
\newcommand{\R}{\mathbb{R}}
\newcommand{\RP}{\mathbb{R}^{+}}
\newcommand{\RN}{\mathbb{R}^{-}}
\renewcommand{\C}{\mathbb{C}}




% matrix symbols
\newcommand{\mA}{\boldsymbol{A}}
\newcommand{\mB}{\boldsymbol{B}}
\newcommand{\mC}{\boldsymbol{C}}
\newcommand{\mD}{\boldsymbol{D}}
\newcommand{\mE}{\boldsymbol{E}}
\newcommand{\mF}{\boldsymbol{F}}
\newcommand{\mG}{\boldsymbol{G}}
\newcommand{\mH}{\boldsymbol{H}}
\newcommand{\mI}{\boldsymbol{I}}
\newcommand{\mJ}{\boldsymbol{J}}
\newcommand{\mK}{\boldsymbol{K}}
\newcommand{\mL}{\boldsymbol{L}}
\newcommand{\mM}{\boldsymbol{M}}
\newcommand{\mN}{\boldsymbol{N}}
\newcommand{\mO}{\boldsymbol{O}}
\newcommand{\mP}{\boldsymbol{P}}
\newcommand{\mQ}{\boldsymbol{Q}}
\newcommand{\mR}{\boldsymbol{R}}
\newcommand{\mS}{\boldsymbol{S}}
\newcommand{\mT}{\boldsymbol{T}}
\newcommand{\mU}{\boldsymbol{U}}
\newcommand{\mV}{\boldsymbol{V}}
\newcommand{\mW}{\boldsymbol{W}}
\newcommand{\mX}{\boldsymbol{X}}
\newcommand{\mY}{\boldsymbol{Y}}
\newcommand{\mZ}{\boldsymbol{Z}}

% vector symbols
\newcommand{\vA}{\boldsymbol{A}}
\newcommand{\vB}{\boldsymbol{B}}
\newcommand{\vC}{\boldsymbol{C}}
\newcommand{\vD}{\boldsymbol{D}}
\newcommand{\vE}{\boldsymbol{E}}
\newcommand{\vF}{\boldsymbol{F}}
\newcommand{\vG}{\boldsymbol{G}}
\newcommand{\vH}{\boldsymbol{H}}
\newcommand{\vI}{\boldsymbol{I}}
\newcommand{\vJ}{\boldsymbol{J}}
\newcommand{\vK}{\boldsymbol{K}}
\newcommand{\vL}{\boldsymbol{L}}
\newcommand{\vM}{\boldsymbol{M}}
\newcommand{\vN}{\boldsymbol{N}}
\newcommand{\vO}{\boldsymbol{O}}
\newcommand{\vP}{\boldsymbol{P}}
\newcommand{\vQ}{\boldsymbol{Q}}
\newcommand{\vR}{\boldsymbol{R}}
\newcommand{\vS}{\boldsymbol{S}}
\newcommand{\vT}{\boldsymbol{T}}
\newcommand{\vU}{\boldsymbol{U}}
\newcommand{\vV}{\boldsymbol{V}}
\newcommand{\vW}{\boldsymbol{W}}
\newcommand{\vX}{\boldsymbol{X}}
\newcommand{\vY}{\boldsymbol{Y}}
\newcommand{\vZ}{\boldsymbol{Z}}

\newcommand{\va}{\boldsymbol{a}}
\newcommand{\vb}{\boldsymbol{b}}
\newcommand{\vc}{\boldsymbol{c}}
\newcommand{\vd}{\boldsymbol{d}}
\newcommand{\ve}{\boldsymbol{e}}
\newcommand{\vf}{\boldsymbol{f}}
\newcommand{\vg}{\boldsymbol{g}}
\newcommand{\vh}{\boldsymbol{h}}
\newcommand{\vi}{\boldsymbol{i}}
\newcommand{\vj}{\boldsymbol{j}}
\newcommand{\vk}{\boldsymbol{k}}
\newcommand{\vl}{\boldsymbol{l}}
\newcommand{\vm}{\boldsymbol{m}}
\newcommand{\vn}{\boldsymbol{n}}
\newcommand{\vo}{\boldsymbol{o}}
\newcommand{\vp}{\boldsymbol{p}}
\newcommand{\vq}{\boldsymbol{q}}
\newcommand{\vr}{\boldsymbol{r}}
\newcommand{\vs}{\boldsymbol{s}}
\newcommand{\vt}{\boldsymbol{t}}
\newcommand{\vu}{\boldsymbol{u}}
\newcommand{\vv}{\boldsymbol{v}}
\newcommand{\vw}{\boldsymbol{w}}
\newcommand{\vx}{\boldsymbol{x}}
\newcommand{\vy}{\boldsymbol{y}}
\newcommand{\vz}{\boldsymbol{z}}


\newcommand{\cA}{\mathcal{A}}
\newcommand{\cB}{\mathcal{B}}
\newcommand{\cC}{\mathcal{C}}
\newcommand{\cD}{\mathcal{D}}
\newcommand{\cE}{\mathcal{E}}
\newcommand{\cF}{\mathcal{F}}
\newcommand{\cG}{\mathcal{G}}
\newcommand{\cH}{\mathcal{H}}
\newcommand{\cI}{\mathcal{I}}
\newcommand{\cJ}{\mathcal{J}}
\newcommand{\cK}{\mathcal{K}}
\newcommand{\cL}{\mathcal{L}}
\newcommand{\cM}{\mathcal{M}}
\newcommand{\cN}{\mathcal{N}}
\newcommand{\cO}{\mathcal{O}}
\newcommand{\cP}{\mathcal{P}}
\newcommand{\cQ}{\mathcal{Q}}
\newcommand{\cR}{\mathcal{R}}
\newcommand{\cS}{\mathcal{S}}
\newcommand{\cT}{\mathcal{T}}
\newcommand{\cU}{\mathcal{U}}
\newcommand{\cV}{\mathcal{V}}
\newcommand{\cW}{\mathcal{W}}
\newcommand{\cX}{\mathcal{X}}
\newcommand{\cY}{\mathcal{Y}}
\newcommand{\cZ}{\mathcal{Z}}


\newcommand{\ldfn}{:\iff}
\newcommand{\edfn}{:=}


\newcommand{\SetZero}{\{0\}}

\newcommand{\etal}{\emph{et al}.}
\newcommand{\ie}{\emph{i.e.,}}
\newcommand{\as}{\emph{a.e.}}
%\renewcommand{\eg}{\emph{e.g.,}}

\newcommand{\eps}{\varepsilon}


%%%%%%%%%%%%%%%%%%%%%% preamble end

%\usepackage[backend=biber,style=numeric,]{biblatex}
%\addbibresource{main-math.bib}

\title{行列式}
\begin{document}
	
	\section{数列}
	
	\section{総和 $\sum$}
	
		\begin{dfn}[総和] 数列 $\Set{a_i}$ の $a_1$ から $a_n$ までの総和 $\sum$ を帰納的に定義する.
			\begin{align}
				\sum_{i=1}^{1} a_{i} &:= a_1 \\
				\sum_{i=1}^{n+1} a_i &:= \left( \sum_{i=1}^{n} a_i \right) + a_{n+1}
			\end{align}
		\end{dfn}
	
		\begin{eg} 総和を括弧 $( \  )$ で加法演算 $+$ の実行順序を表すと
			\begin{equation}
				\sum_{i=1}^n a_i = ((\dots((a_1) + a_2)+ \dots )+ a_{n-1}) + a_n
			\end{equation}
			となる.括弧の役割は,加法演算の実行順序を表す.実行順序を加法 $+$ への下付き数字で表すことにより,括弧をはずしてもよい.すなわち
			\begin{equation}
				a_1 +_{1} a_2 +_{2} \dots +_{n-2} \ \ a_{n-1} +_{n-1} a_n = \ ((\dots((a_1) + a_2)+ \dots )+ a_{n-1}) + a_n
			\end{equation}
			が成り立っている.
			
			例えば,加法の結合法則は,.
			\begin{equation}
				a + (b + c) = (a + b) + c
			\end{equation}
			であるから,この加法の実行順序を表すと
			\begin{equation} 
				a +_2 b +_1 c= a +_1 b +_2 c
			\end{equation}
			今,置換 $\sigma$ を
			\begin{equation}
				\sigma = 
				\begin{pmatrix}
					1 & 2 \\
					2 & 1 
				\end{pmatrix}
			\end{equation}
			とすれば,結合法則は
			\begin{equation} 
				a +_{\sigma(1)} \ \ b +_{\sigma(2)} \ \ c= a +_1 b +_2 c
			\end{equation}
			と表すことができる.
			
			$n$ 項の結合法則は隣り合う加法演算の順序を交換することにより,加法演算の順序が左から順番に演算することに等しいことができる.すなわち,
			\begin{equation}
				a_1 \ +_{\sigma(1)} \ a_2 \ +_{\sigma(2)}  \dots +_{\sigma(n-1)} \ a_n = a_1 \ +_1 \ a_2 \ +_2 \dots +_{n-1} \ a_n 
			\end{equation}
			が成り立つ.これは,任意の置換が隣接互換の積で表せることで\part{title}
		\end{eg}
	
		\begin{thm}[一般の結合法則] $n$ 項の加法は,加法の順番によらず,一定の値になる.すなわち,
			\begin{equation}
				a_1 \ +_{\sigma(1)} \ a_2 \ +_{\sigma(2)}  \dots +_{\sigma(n-1)} \ a_n = a_1 \ +_1 \ a_2 \ +_2 \dots +_{n-1} \ a_n 
			\end{equation}
		\end{thm}
	
		\begin{thm}[一般の交換法則]
		\end{thm}
	
		\begin{dfn} 自然数の集合 $\N_1^n$ から集合 $\Lambda$ への全単射 $\varphi :  \N_1^n \map \Lambda$ が存在するとき, $\Lambda$ を添字集合とする数列 $\Set{a_\lambda}_{\lambda \in \Lambda} $ の総和 $\sum_{\lambda \in \Lambda} a_{\lambda}$ を次のように定義する.
			\begin{equation}
				\sum_{\lambda \in \Lambda} a_{\lambda} := \sum_{i=1}^n a_{\varphi(i)}
			\end{equation}
		\end{dfn}
	
	\section{行列式の定義}
	
		\begin{dfn} 行列 $\mA = (\va_1, \dots,\va_n) = (a_{i j})$ とするとき,この行列 $\mA$ の行列式 $\det(\mA)$ を次のように定義する.
			\begin{equation}
				\det(\mA) := \sum_{\sigma \in S_n} \sgn(\sigma) \ a_{1 \, \sigma(1)} \dots a_{n \, \sigma(n)}
			\end{equation}
		
			行列式は,次のように表すことがある.
			\begin{equation}
				\det(\va_1,\dots,\va_n), \det(a_{i j}), \rvert A \lvert,  \rvert \va_1, \dots, \va_n \lvert, \rvert a_{i j} \lvert,
				\begin{vmatrix}
					a_{1 1} & a_{1 2} & \dots & a_{1 n} \\
					a_{2 1} & a_{2 2} & \dots & a_{2 n } \\
					           &             & \dots &              \\
					a_{n 1} & a_{n 2} & \dots & a_{n n}					           
				\end{vmatrix}
			\end{equation}
		\end{dfn}
	

		
	\section{行列式の双線形性}
		\begin{thm} 双線形性
			\begin{enumerate}
				\item a
					\begin{equation}
						\det(\va_1, \dots, \va_j' + \va_j'', \dots, \va_n) = \det(\va_1, \dots, \va_j', \dots, \va_n) + \det(\va_1, \dots, \va_j'', \dots, \va_n)  
					\end{equation}
				
					\begin{equation}
						\begin{vmatrix}
							a_{1 1} & \dots & a_{1 j}' + a_{1 j}''  & \dots & a_{1 n} \\
							           & \dots &   \dots                & \dots &           \\
							a_{n 1} & \dots & a_{n j}' + a_{n j}'' & \dots & a_{n n}
						\end{vmatrix} 
						=
						\begin{vmatrix}
							a_{1 1} & \dots & a_{1 j}' & \dots & a_{1 n} \\
							& \dots &   \dots                & \dots &           \\
							a_{n 1} & \dots & a_{n j}' & \dots & a_{n n}
						\end{vmatrix} 
						+
						\begin{vmatrix}
							a_{1 1} & \dots & a_{1 j}''  & \dots & a_{1 n} \\
							& \dots &   \dots                & \dots &           \\
							a_{n 1} & \dots &  a_{n j}'' & \dots & a_{n n}
						\end{vmatrix}
					\end{equation}
				\item b
					\begin{equation}
						\det(\va_1, \dots, \lambda \va_j, \dots, \va_n) = \lambda \det(\va_1, \dots, \va_j, \dots, \va_n)
					\end{equation}
					\begin{equation}
						\begin{vmatrix}
							a_{1 1} & \dots & \lambda a_{1 j} & \dots & a_{1 n} \\
							& \dots &   \dots                & \dots &           \\
							a_{n 1} & \dots & \lambda a_{n j}  & \dots & a_{n n}
						\end{vmatrix} 
						= \lambda
						\begin{vmatrix}
							a_{1 1} & \dots & a_{1 j} & \dots & a_{1 n} \\
							& \dots &   \dots                & \dots &           \\
							a_{n 1} & \dots & a_{n j} & \dots & a_{n n}
						\end{vmatrix} 
					\end{equation}
			\end{enumerate}
		\end{thm}

		\begin{thm} $\tau \in S_n$
			\begin{equation}
				\det(\va_{\tau(1)}, \dots, \va_{\tau(j)}, \dots, \va_{\tau(n)}) = \sgn(\tau) \det(\va_1, \dots, \va_j, \dots, \va_n)
			\end{equation}
			\begin{equation}
				\begin{vmatrix}
					a_{1 \tau(1)} & \dots & \lambda a_{1 \tau(j)} & \dots & a_{1 \tau(n)} \\
					& \dots &   \dots                & \dots &           \\
					a_{n \tau(1)} & \dots & \lambda a_{n \tau(j)}  & \dots & a_{n \tau(n)}
				\end{vmatrix} 
				= \sgn(\tau)
				\begin{vmatrix}
					a_{1 1} & \dots & a_{1 j} & \dots & a_{1 n} \\
					& \dots &   \dots                & \dots &           \\
					a_{n 1} & \dots & a_{n j} & \dots & a_{n n}
				\end{vmatrix} 
			\end{equation}
		\end{thm}

		\begin{thm}
			\begin{equation}
				\det({}^t\!\mA) = \det(\mA)
			\end{equation}
		\end{thm}

	
	\section{行列式の展開}
		\begin{dfn}
			\begin{equation}
				\mA_{i j} = \begin{pmatrix}
						a_{1 1} & a_{1 2} & \dots & a_{1 j} & \dots & a_{1 n} \\
						a_{2 1} & a_{2 2} & \dots & a_{2 j} & \dots & a_{2 n} \\
						            &            & \dots &      &     & \\
					   a_{i 1} & a_{i 2} & \dots & a_{i j} & \dots & a_{i n} \\
						            &            & \dots &      &     & \\
						a_{n 1} & a_{n 2} & \dots & a_{n j} & \dots & a_{n n} \\          
					\end{pmatrix}
			\end{equation}
		\end{dfn}
	
		\begin{thm}
			\begin{equation}
				\det(\mA) = a_{i 1} \det(\mA_{1 i}) + \dots + a_{i j} \det(\mA_{i j}) + \dots + a_{i n} \det(\mA_{i n})
			\end{equation}
		\end{thm}

\end{document}