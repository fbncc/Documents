\documentclass[leqno,autodetect-engine, dvipdfmx-if-dvi,ja=standard]{bxjsarticle}
\input{header.tex}
\title{行列式}
\begin{document}
	
	\section{行列式の定義}
	
		\begin{dfn} 行列 $\mA = (\va_1, \dots,\va_n) = (a_{i j})$ とするとき,この行列 $\mA$ の行列式 $\det(\mA)$ を次のように定義する.
			\begin{equation}
				\det(\mA) := \sum_{\sigma \in S_n} \sgn(\sigma) \ a_{1 \, \sigma(1)} \dots a_{n \, \sigma(n)}
			\end{equation}
		
			行列式は,次のように表すことがある.
			\begin{equation}
				\det(\va_1,\dots,\va_n), \det(a_{i j}), \rvert A \lvert,  \rvert \va_1, \dots, \va_n \lvert, \rvert a_{i j} \lvert,
				\begin{vmatrix}
					a_{1 1} & a_{1 2} & \dots & a_{1 n} \\
					a_{2 1} & a_{2 2} & \dots & a_{2 n } \\
					           &             & \dots &              \\
					a_{n 1} & a_{n 2} & \dots & a_{n n}					           
				\end{vmatrix}
			\end{equation}
		\end{dfn}
	

		
	\section{行列式の双線形性}
		\begin{thm} 双線形性
			\begin{enumerate}
				\item a
					\begin{equation}
						\det(\va_1, \dots, \va_j' + \va_j'', \dots, \va_n) = \det(\va_1, \dots, \va_j', \dots, \va_n) + \det(\va_1, \dots, \va_j'', \dots, \va_n)  
					\end{equation}
				
					\begin{equation}
						\begin{vmatrix}
							a_{1 1} & \dots & a_{1 j}' + a_{1 j}''  & \dots & a_{1 n} \\
							           & \dots &   \dots                & \dots &           \\
							a_{n 1} & \dots & a_{n j}' + a_{n j}'' & \dots & a_{n n}
						\end{vmatrix} 
						=
						\begin{vmatrix}
							a_{1 1} & \dots & a_{1 j}' & \dots & a_{1 n} \\
							& \dots &   \dots                & \dots &           \\
							a_{n 1} & \dots & a_{n j}' & \dots & a_{n n}
						\end{vmatrix} 
						+
						\begin{vmatrix}
							a_{1 1} & \dots & a_{1 j}''  & \dots & a_{1 n} \\
							& \dots &   \dots                & \dots &           \\
							a_{n 1} & \dots &  a_{n j}'' & \dots & a_{n n}
						\end{vmatrix}
					\end{equation}
				\item b
					\begin{equation}
						\det(\va_1, \dots, \lambda \va_j, \dots, \va_n) = \lambda \det(\va_1, \dots, \va_j, \dots, \va_n)
					\end{equation}
					\begin{equation}
						\begin{vmatrix}
							a_{1 1} & \dots & \lambda a_{1 j} & \dots & a_{1 n} \\
							& \dots &   \dots                & \dots &           \\
							a_{n 1} & \dots & \lambda a_{n j}  & \dots & a_{n n}
						\end{vmatrix} 
						= \lambda
						\begin{vmatrix}
							a_{1 1} & \dots & a_{1 j} & \dots & a_{1 n} \\
							& \dots &   \dots                & \dots &           \\
							a_{n 1} & \dots & a_{n j} & \dots & a_{n n}
						\end{vmatrix} 
					\end{equation}
			\end{enumerate}
		\end{thm}

		\begin{thm} $\tau \in S_n$
			\begin{equation}
				\det(\va_{\tau(1)}, \dots, \va_{\tau(j)}, \dots, \va_{\tau(n)}) = \sgn(\tau) \det(\va_1, \dots, \va_j, \dots, \va_n)
			\end{equation}
			\begin{equation}
				\begin{vmatrix}
					a_{1 \tau(1)} & \dots & \lambda a_{1 \tau(j)} & \dots & a_{1 \tau(n)} \\
					& \dots &   \dots                & \dots &           \\
					a_{n \tau(1)} & \dots & \lambda a_{n \tau(j)}  & \dots & a_{n \tau(n)}
				\end{vmatrix} 
				= \sgn(\tau)
				\begin{vmatrix}
					a_{1 1} & \dots & a_{1 j} & \dots & a_{1 n} \\
					& \dots &   \dots                & \dots &           \\
					a_{n 1} & \dots & a_{n j} & \dots & a_{n n}
				\end{vmatrix} 
			\end{equation}
		\end{thm}

		\begin{thm}
			\begin{equation}
				\det({}^t\!\mA) = \det(\mA)
			\end{equation}
		\end{thm}

	
	\section{行列式の展開}
		\begin{dfn}
			\begin{equation}
				\mA_{i j} = \begin{pmatrix}
						a_{1 1} & a_{1 2} & \dots & a_{1 j} & \dots & a_{1 n} \\
						a_{2 1} & a_{2 2} & \dots & a_{2 j} & \dots & a_{2 n} \\
						            &            & \dots &      &     & \\
					   a_{i 1} & a_{i 2} & \dots & a_{i j} & \dots & a_{i n} \\
						            &            & \dots &      &     & \\
						a_{n 1} & a_{n 2} & \dots & a_{n j} & \dots & a_{n n} \\          
					\end{pmatrix}
			\end{equation}
		\end{dfn}
	
		\begin{thm}
			\begin{equation}
				\det(\mA) = a_{i 1} \det(\mA_{1 i}) + \dots + a_{i j} \det(\mA_{i j}) + \dots + a_{i n} \det(\mA_{i n})
			\end{equation}
		\end{thm}

\end{document}