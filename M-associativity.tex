\documentclass[leqno,autodetect-engine, dvipdfmx-if-dvi,ja=standard]{bxjsarticle}
\input{header.tex}
\title{結合法則}
\begin{document}
	\section{総和 $\sum$ について}

	\begin{dfn} 二項演算 加法 $(+)$ は,結合法則を満足しているとする.
		\begin{align}
			(a_1 + a_2) + a_3 &= a_1 + (a_2 + a_3)  & \text{結合法則}
		\end{align}
	\end{dfn}

	結合法則は,加法をどの順番に行ってもその結果は変わらないことを意味している.
	加法の記号 $+$ の下付き文字で加法の順番を表すと
	\[
	(a_1 + a_2) + a_3
	\]
	は
	\[
	a_1 +_1 a_2 +_2 a_3
	\]
	と表すことができる.これは,次の手順で計算を行うことになる.
	\begin{enumerate}
		\item まず,$a_1 +_1 a_2$ を計算し,結果を $A$ とする.
		\item 次に,$A +_2 a_3$ を計算し,この結果を $(a_1 + a_2) + a_3$ の値 $S_1$ とする.
	\end{enumerate}

	次に
	\[
	a_1 + (a_2 + a_3)
	\]
	は,
	\[
	a_1 +_2 + a_2 +_1 a_3
	\]
	を表している.これは,次の手順で計算を行うことになる.
	\begin{enumerate}
		\item まず,$a_2 +_1 a_3$ を計算し,結果を $B$ とする.
		\item 次に,$a_1 +_2 B$ を計算し,この結果を $a_1 + (a_2 + a_3)$ の値 $S_2$ とする.
	\end{enumerate}

	一般に $n$ 項の加法は,$n-1$ 個の $+$ があるから,加法の順番は,各 $+$ に $1,2,\dots,n-1$ の数字を割り振った順列になる.

	$n=3$ の場合,加法の順番は,$+$ の個数が2になるから,$\Set{1,2}$ の順列となる.すなわち,
	\begin{align}
	 1.2 \\
	 2,1
	\end{align}
	だから,加法の順番,次の2通りがある.
	\begin{align}
		a_1 +_1 a_2 +_2 a_3 &= (a_1 + a_2) + a_3 \\
		a_1 +_2 a_2 +_1 a_3 &= a_1 + (a_2 + a_3)
	\end{align}

	3項の加法は加法の順番にかかわらず,その結果は等しいことを主張する式は,次のようになる.
	\[
		a_1 +_1 a_2 +_2 a_3 = a_1 +_2 a_2 +_1 a_3
	\]
	すなわち,これを括弧式で表せば,次の式となる.
	\[
		(a_1 + a_2) + a_3 = a_1 + (a_2 + a_3)
	\]
	結合法則は,加法の順番にかかわらず,その結果は等しいことを意味している.

	さて,この結合法則を使えば,3項以上の加法についても加法の順番にかかわらず,その結果が等しいことが証明できる.

	まず,最初に3項以上の加法を定義する.
	\begin{dfn} $n (\ge 3)$ 個の元の和を次のように帰納的に定義する.
		\begin{align}
			a_1 + a_2 + a_3 &= (a_1 + a_2) + a_3 \\
			a_1 + a_2 + \dots + a_n &= (a_1 + a_2 + \dots + a_{n-1}) + a_n
		\end{align}
	\end{dfn}


	\begin{nb} $a_1 + a_2 + \dots + a_n$ は括弧式では次のようになる.
		\begin{equation}
		a_1 + a_2 + \dots + a_n = (\cdots(a_1 + a_2) + \dots + a_{n-2})+ a_{n-1}) + a_n
		\end{equation}
		また,加法の順番を下付き文字で表せば,
		\begin{equation}
		a_1 + a_2 + \dots + a_n = a_1 +_1 a_2 +_2 \dots +_{n-2} a_{n-1} +_{n-1} a_n
		\end{equation}
		となる.

		4項の場合は,2組の括弧で加法の順番を表すことができる.

		3項の場合は,1組の括弧で加法の順番を表すことができる.
	\end{nb}

	\begin{nb}加法の順番を括弧で表す.
		\begin{align}
			a_1 +_1 a_2 +_2 a_3 +_3 a_4 &=  ((a_1 +_1 a_2) +_2 a_3) +_3 a_4 \\
			a_1 +_1 a_2 +_3 a_3 +_2 a_4 &= (a_1 +_1 a_2) +_3 (a_3 +_2 a_4) \\
			a_1 +_2 a_2 +_1 a_3 +_3 a_4 &= (a_1 +_2 (a_2 +_1 a_3)) +_3 a_4 \\
			a_1 +_2 a_2 +_3 a_3 +_1 a_4 &= (a_1 +_2 a_2) +_3 (a_3 +_1 a_4) \\
			a_1 +_3 a_2 +_1 a_3 +_2 a_4 &= a_1 +_3 ((a_2 +_1 a_3) +_2 a_4) \\
			a_1 +_3 a_2 +_2 a_3 +_1 a_4 &= a_1 +_3 (a_2 +_2 (a_3 +_1 a_4))
		\end{align}
	\end{nb}

	\begin{thm} $n$ 項 $(n \ge 3)$ の加法は $n-2$ 組の括弧で加法の順番を表すことができる.
	\end{thm}
	\begin{proof} $n$ の帰納法で示す.
		\begin{enumerate}
			\item $n=3$ の場合,$1$ 組の括弧で加法の順番を表すことができる.
				\begin{align}
					a_1 +_1 a_2 +_2 a_3 &= (a_1 + a_2) + a_3 \\
					a_1 +_2 a_2 +_1 a_3 &= a_1 + (a_2 + a_3)
				\end{align}
			\item $n$ のとき命題が成り立つとして,$n+1$ の場合に命題が成り立つことを示す.

				$i_k=1$ となる$k$ が存在する.このとき,$A$ を
				\[
				A := a_k +_{i_k} + a_{k+1}
				\]
				と置く.

				さて,
				\[
					a_1 +_{i_1} + a_2 + \dots +_{k-1} a_k +_{i_k} + a_{k+1} +_{i_{k+1}} \dots +_n a_{n+1}
				\]
				は
				\[
					a_1 +_{i_1} + a_2 + \dots +_{k-1} A +_{i_{k+1}} \dots +_n a_{n+1}
				\]
				と表すことができる.この式は,$n$ 項の式であるから,帰納法の仮定により,$n-2$ 組の括弧で加法の順序を表すことができる.
				したがって,$A$ を囲む1組の括弧,つまり,$(a_k +_{i_k} + a_{k+1})$ 囲む括弧と 2番目から$n$番目の加法を表す $n-2$ 組の括弧で,元の式の加法の順序を表すことができた.
				\[
					a_1 +_{i_1} + a_2 + \dots +_{k-1} (a_k +_{i_k} + a_{k+1}) +_{i_{k+1}} \dots +_n a_{n+1}
				\]
		\end{enumerate}
	以上で示すことができた.
	\end{proof}

	\begin{thm}[一般結合法則] $n \ge 3$ 項の和に関して,$i_1,i_2,\dots,i_{n-1}$ を $1,2,\dots,n-1$ の順列とする.
		このとき,次の式が成り立つ.
		\[
		a_1 +_{i_1} a_2 +_{i_2} \dots +_{n-1} a_n = a_1 +_1  a_2 +_2 \dots +_{n-1} a_n
		\]
	\end{thm}
	\begin{proof} $n$ の帰納法で証明する.
		\begin{enumerate}
			\item $n=3$ のとき,$i_1$ と $i_2$ の組み合わせは,
				\begin{gather}
					i_1=1 , i_2=2 \\
					i_1=2, i_2=1
				\end{gather}
				の2つの組み合わせがある.

				まず,$i_1=1,i_2=2$ の場合は,
				\[
				a_1 +_{i_1} + a_2 +_{i_2} = a_1 +_1 + a_2 +_2 a_3
				\]
				であり,命題が成り立つ.

				次に,$i_1=2,i_2=1$ の場合は,
				\[
				a_1 +_{i_1} + a_2 +_{i_2} + a_3= a_1 +_2 + a_2 +_1 a_3
				\]
				だから,結合法則により,
				\begin{align}
					a_1 +_{i_1} + a_2 +_{i_2} +a_3 &= a_1 +_2 + a_2 +_1 a_3 \\
					&= a_1 +_1 a_2 +_2 a_3
				\end{align}
				となるから,命題が成り立つ.
			\item $n$ の場合を仮定し,$n+1$ の場合にも命題が成り立つことを示す.すなわち,$i_1,i_2,\dots,i_n$ を $1,2,\dots,n$ の順列とする.このとき,
				\begin{equation}
					a_1 +_{i_1} a_2 +_{i_2} \dots +_{i_n} a_{n+1}  = a_1 +_1  a_2 +_2 \dots +_n a_{n+1}
				\end{equation}
				を示せばよい.

				さて,$i_k = 1$ となる $k$ が存在する.
				この $k$ について,
				\begin{enumerate}
					\item $k=1$ の場合, つまり,$i_1=1$ で,
						\[
								a_1 +_{i_1} a_2 +_{i_2} \dots +_{i_n} a_{n+1}  = a_1 +_1 a_2 +_{i_3} \dots +_{i_n} a_{n+1}
						\]
						ここで,$a_1 +_1 a_2$ を $A$ と置くと,
						\[
							 a_1 +_1 a_2 +_{i_2} \dots +_{i_n} a_{n+1} = A +_{i_2} \dots +_{i_n} a_{n+1}
						\]
						となる.右辺の加法の個数は $n$ 個となる.帰納法の仮定より,右辺は,
						\[
							A +_2 a_3 +_3 + \dots +_{n}
						\]
						と等しい.$A$ を $a_1 +_1 + a_2$ に戻すと,
						\[
							a_1 +_1 a_2 +_2 a_3 +_3 + \dots +_{n}
						\]
						となる.すなわち,命題が成り立つことがわかる.
					\item $1<k \le n$ の場合,つまり,$i_k=1$ で,
						\[
						a_1 +_{i_1} a_2 +_{i_2} \dots +_{i_{k-1}} a_k +_{i_k} + a_{k+1} +_{i_{k+1}} \dots +_{i_n} a_{n+1}
						\]
						は
						\[
						a_1 +_{i_1} a_2 +_{i_2} \dots +_{i_{k-1}} a_k +_1 + a_{k+1} +_{i_{k+1}} \dots +_{i_n} a_{n+1}
						\]
						である.ここで,$a_k +_1 a_{k+1}$ を $A$ と置くと,
						\[
						a_1 +_{i_1} a_2 +_{i_2} \dots +_{i_{k-1}} A +_{i_{k+1}} \dots +_{i_n} a_{n+1}
						\]
						となり,加法の個数は $n$ 個となる.帰納法の仮定より,
						\[
						a_1 +_1 a_2 +_2 \dots +_{k-1} A +_{k+1} \dots +_{n} a_{n+1}
						\]
						と等しい.

						さて,$a_1 +_1 a_2 +_2 \dots +_{k-1} A$ について, $a_1 +_1 + a_2 +_3 \dots +_{k-2} a_{k-1}$ を $B$ と置くと,
						\begin{align}
						a_1 +_1 a_2 +_2 \dots +_{k-2} a_{k-1} +_{k-1} A &= B +_{k-1} A \\
						&= B +_{k-1} (a_k +_1 a_{k+1} ) \\
						&= B +_1 + a_k +_2 a_{k+1}
						\end{align}
						最後の式の $B$ を下に元に戻して,加法の順序を順番に付け替えると,
						\[
						B +_1 + a_k +_2 a_{k+1} = a_1 +_1 a_2 +_2 + \dots +_{k-1} a_k +_k a_{k+1}
						\]
						この結果を,
						\[
						a_1 +_1 a_2 +_2 \dots +_{k-1} A +_{k+1} \dots +_{n} a_{n+1}
						\]
						に戻すと
						\[
						a_1 +_1 a_2 +_2 \dots +_{k-1} a_k +_k a_{k+1} +_{k+1} \dots +_{n} a_{n+1}
						\]
						となり,命題が成り立つことがわかる.
				\end{enumerate}
				これで証明ができたわけである.
		\end{enumerate}
	\end{proof}




	\begin{thm}[一般結合法則] $n \ge 3$ 項の和に関して,$1 \le m \le n $ 任意の $m$ について次の式が成り立つ.
		\begin{equation}
			(a_1 + \dots + a_m) + (a_{m+1} + \dots + a_n) = a_1 + \dots + a_m + a_{m+1} + \dots + a_n
		\end{equation}
		この式は,加法の順番を表す記号 $+_i$ を使うと,次のように表すことができる.

		$i_1,i_2,\dots,i_n$ を $1,2,\dots,n$ の順列とすると,次の式が成り立つ.
		\[
		a_1 +_{i_1} a_2 +_{i_3} \dots + a_{i_n} = a_1 +_1  a_2 +_2 \dots +_{n-1} a_n
		\]
	\end{thm}
	\begin{proof} 証明 1

		$n$ の帰納法で証明する.
		\begin{enumerate}
		\item $n=3$ のとき

			\begin{align}
				a_1 +_{1}  a_2 +_{2} a_3 = (a_1 + a_2) + a_3 \\
				a_1
			\end{align}

		\item $n$ のとき正しいと仮定し,$n'$ のとき正しいことを証明する.

		$m+1=n$ なら定義から明らか.$m+1 <n$ とする.
		\begin{align}
			(a_1 + \dots + a_m) + (a_{m+1} + \dots + a_n) &= (a_1 + \dots + a_m) + ((a_{m+1} + \dots + a_{n-1}) + a_n) \\
			&= ((a_1 + \dots + a_m) + (a_{m+1} + \dots + a_{n-1})) + a_n \\
			&= (a_1 + \dots + a_m + a_{m+1} + \dots + a_{n-1}) + a_n \\
			&= a_1 + \dots + a_m + a_{m+1} + \dots + a_n
		\end{align}

		\end{enumerate}
	\end{proof}

	\begin{dfn} 加法は次の交換法則を満足する.
		\begin{align}
			a_1 + a_2 &= a_2 + a_1 & \text{交換法則}
		\end{align}
	\end{dfn}

	\begin{thm}[一般交換法則] $i_1,i_2, \dots, i_n$ を任意の $1,2,\dots,n$ の順列とする.このとき,次の式が成り立つ.
		\begin{equation}
			a_{i_1} + a_{i_2} + \dots + a_{i_n} = a_1 + a_2 + \dots + a_n
		\end{equation}
	\end{thm}
	\begin{proof} $i_n=n$ ならば帰納法の仮定より明らか.$i_n \not= n$ とし,$i_k =n$ とする.
		\begin{align}
			a_{i_1} + a_{i_2} + \dots + a_{i_n} &= (a_{i_1} + \dots + a_{k-1}) + a_n + (a_{k+1} + \dots + a_n) \\
			&= (a_{i_1} + \dots + a_{k-1}) + (a_{k+1} + \dots + a_n) + a_n \\
			&= (a_1 + \dots + a_{n-1}) + a_n \\
			&= a_1 + \dots + a_n
		\end{align}
	\end{proof}

\end{document}
